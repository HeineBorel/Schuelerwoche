\documentclass[a4paper,10pt]{article}

\usepackage{mystyle}
\usepackage{amsmath}
\usepackage{amsthm}
\usepackage{mathrsfs} %Für das curly L der linearen Operatoren
\usepackage{empheq}


\begin{document}

\begin{Aufg}
 In der Vorlesung wurde definiert was ein Abstandmaß oder Metrik ist:
 \begin{enumerate}[(i)]
  \item $d(x, y) \geq 0 $ und $d(x,y) =0$ genau dann, wenn $x=y$
  \item $d(x,y) = d(y,x)$
  \item $d(x,y) \leq d(x,z) + d(z,y)$ ``Dreiecksungleichung''
 \end{enumerate}
 
 Wir wollen nun nachrechnen ob es überhaupt solche Abbildungen gibt,
 oder ob wir schon solche Abbildungen kennen. 
 \begin{enumerate}
  \item Prüfe ob die Abbildung $(x,y) \mapsto |x-y|$ ein Abstandmaß auf $\R$ ist.
  \item Auf $\R^2$ definiert man $|(x_1, x_2)| \coloneqq \sqrt{x_1 ^2 + x_2 ^2}$.
	Ist dann die Abbildung $(x,y) \mapsto |x-y|$ ein Abstandmaß auf $\R^2$?
  \item Nun wollen wir eine Metrik auf beliebigen Mengen definieren. Sei dazu $X$
	eine Menge. Definiere $\Phi$ durch
  \begin{gather*}
   (a,b)\mapsto \begin{cases}
            1  \text{ wenn } a=b \\
            0  \text{ wenn } a\neq b.
            \end{cases}
  \end{gather*}
 Zeige, dass die Abbildung $(a,b) \mapsto \Phi(a,b)$ ein Abstandmaß auf $X$ definiert. 
 Diese Metrik ist die sogenannte diskrete Metrik.
  
 \end{enumerate}

\end{Aufg}
 
\begin{Aufg}
Sei $d(x,y)$ ein Abstandmaß auf einer Menge $X$. Zeige, dass die Abbildung
 $(x,y) \mapsto \frac{d(x,y)}{1 + d(x,y)}$ ein Abstandmaß ist.
\end{Aufg}


\begin{Aufg}
Im Beweis des Banachschen Fixpunktsatzes wurde eine Formel über die geometrische Reihe
benutzt. Wir wollen diese Aussage beweisen.

Eine geometrische Reihe ist eine Summe der Form $\sum_{k=0}^N p^k$. Sei nun \linebreak $0<p<1$.
Zeige, dass $\sum_{k=0}^N p^k=  \frac{1- p^{N+1}}{1 - p}$.

Betrachte dazu die Ausdrücke 
\begin{gather*}
 \sum_{k=0}^N p^k= s = 1 + p + p^2 + p^3 + \dots + p^N, \text{ und }\\
 ps =p +p^2 +p^3 + p^4 + \dots + p^{N+1}.
\end{gather*}

Falls nun $N$ gegen $\infty$ geht, so konvergiert der Term $\frac{1- p^{N+1}}{1 - p}$
gegen $\frac{1}{1 - p}$, also $\sum_{k=0}^\infty p^k=  \frac{1}{1 - p}$.
Warum ist die Annahme $0<p<1$ wichtig? Würde die Formel in diesen Fällen Sinn ergeben?
\end{Aufg}


\begin{Aufg}
 Sei $g:\R \to \R$ beliebig oft differenzierbar mit $g(\alpha) = \alpha$, und
 $|g'(\alpha)| <1$. 
 
 Zeige, dass es ein $\epsilon > 0$ sodass $g$ das Intervall 
 $I=\left[ \alpha - \epsilon , \alpha + \epsilon \right]$ auf sich selbst abbildet, d.h. 
 $g(x)$ ist ein Element von $I$ falls $x$ ein Element von $I$ war.
 
 Beweise nun, dass es ein $0<L<1$ gibt, so dass
 \[
  |g(x) - g(y)| \leq L|x-y|, \text{ f"ur alle } x,y \text{ in } I.
 \]

 Folgere nun, dass die Folge $x_{n+1} = g(x_n)$ f"ur $x_0$ in $I$ gegen $\alpha$
 konvergiert.

\end{Aufg}


\begin{Aufg}
 Sei $f: \R \to \R$ eine beliebig differenzierbare Funktion, und $\alpha$ eine Nullstelle von $f$,
 mit $f'(\alpha) \neq 0$.
 \begin{enumerate}[(a)]
  \item Zeige: $\alpha$ ist ein Fixpunkt der Funktion $g(x)= x - \frac{f(x)}{f'(x)}$.
  \item Beweise, dass ein $\epsilon > 0$ und $0<L<1$ existiert, so dass $g$ Lipschitz 
  mit Konstante $L$ auf 
  $I=\left[ \alpha - \epsilon , \alpha + \epsilon \right]$ ist, d.h. 
  $|g(x) - g(y)| \leq L|x-y|$ f"ur alle $x,y$ in $I$.
  
  \textit{
  Hinweis: Ihr k"onnt ohne Beweis den Mittelwertsatz benutzen: }
  
\textit{  F"ur eine differenzierbare}
  $f:\left[ a, b \right] \to \R$ \textit{existiert ein }$\xi$ in $\left[ a, b \right]$,
  \textit{so dass}
  $f(b) - f(a) = f'(\xi)(b-a)$.
  \item Folgere, dass $g$ eine Selbstabbildung des Intervalls
  $I=\left[ \alpha - \epsilon , \alpha + \epsilon \right]$ ist.
  \item Zeige, dass das Newton-Verfahren konvergiert.
 \end{enumerate}

\end{Aufg}



\begin{Aufg}

Nun wollen wir ein Beispiel für eine Fixpunktiteration betrachten, das sogenannte
Newton-Verfahren. Sei dazu $f:\R \to \R$
eine zweimal stetig differenzierbare Funktion mit einer Nullstelle $\alpha$, also $f(\alpha) =0$,
sodass die Ableitung bei $x_0$ nicht gleich 0 ist. Dann definieren wir die iterierte Folge,
\[
 x_{n+1} = x_n - \frac{f(x_n)}{f'(x_n)}, 
\]
mit einem Startwert $x_0$. Um die Rechnungen zu vereinfachen, nehmen wir an, dass $x_0$ nahe bei
der Nullstelle ist, das heißt $|x_0 - \alpha|<1$. 
Weiter können wir $f$ gut durch seine Ableitung annähern:
\[
 f(a) = f(x_n) + f'(x_n)(a-x_n) + C(a-x_n)^2 \text{ für ein } C \in \R.
\]

Um zu beweisen, dass die Folge $x_n$ gegen $\alpha$ konvergiert, betrachte die vorherige
Gleichung mit $a = \alpha$. Dann benutze, dass $f'(x_n) \neq0$, und die Definition der
Folge $x_n$. Finde damit eine iterative Formel für die Folge des Abstands zu der Nullstelle
\[
 \epsilon_n \coloneqq \alpha - x_{n}.
\]
Benutze nun, dass $|x_0 - \alpha|<1$ und damit $\epsilon_n$ gegen 0 konvergiert. Damit konvergiert
$x_n$ gegen $\alpha$.

\end{Aufg}


\begin{Aufg}
Nimm zwei Blatt Papier und lege sie aufeinander. Wir wollen eigentlich eine Selbstabbildung
eines Blatts auf sich selbst betrachten, doch ist es einfacher dafür ein zweites Blatt herzunehmen.
Zerknülle nun das oberere Blatt und lege den Papierball wieder auf das andere Blatt. Nun drücke den 
Ball flach auf das untere Blatt. Jetzt haben wir eine (Selbst)Abbildung eines Blattes auf sich selbst.
Hat diese Abbildung einen Fixpunkt? Ist diese Abbildung eine Kontraktion?
 
\end{Aufg}


\begin{Aufg}
 
 Zeige, dass alle Vorraussetzungen des Brouwerschen Fixpunktsatzes gebraucht werden. Finde also 
 Gegenbeispiele in denen jeweils eine oder mehrere Vorraussetzungen nicht erf"ullt sind.
 
\end{Aufg}

\begin{Aufg}
 Sei $f: \R \to \R$ eine Abbildung, so dass $f^2$ eine Kontraktion ist. 
 
 Zeige $f$ ist eine Kontraktion, und die Fixpunkte von $f$ und $f^2$ stimmen 
 "uberein.
 
\end{Aufg}

\end{document}

