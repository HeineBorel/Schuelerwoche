\documentclass[a4paper,10pt]{article}

\usepackage{mystyle}
\usepackage{amsmath}
\usepackage{amsthm}
\usepackage{mathrsfs} %Für das curly L der linearen Operatoren
\usepackage{empheq}


\begin{document}
\begin{center}
{\Large \textbf{Aufgaben zu dem Brouwerschen Fixpunktsatz}}
 
\end{center}






\begin{Aufg}
 
 Zeige, dass alle Vorraussetzungen des Brouwerschen Fixpunktsatzes gebraucht werden. Finde also 
 Gegenbeispiele in denen jeweils eine oder mehrere Vorraussetzungen nicht erf"ullt sind.
 
\end{Aufg}

\begin{proof}\leavevmode
 \begin{enumerate}
  \item Stetigkeit: Drehung + einen Punkt mit dem Mittelpunkt vertauschen
  \item Zusammenhang der zugrunde liegenden Menge (also nicht $D^2$): Zwei $D^2$ wobei
  die Abbildung einfach die Mengen vertauscht
 \end{enumerate}

\end{proof}


\begin{Aufg}
 
 Sei $\Phi:D^2 \to X$ eine stetige bijektive Abbildung mit stetigem Inversen $\Phi^{-1}:X \to D^2$.
 Das heißt $X$ und $D^2$ sind "uber die Abbildung $\Phi$ mit einander zu identifizieren:
 Jedem Punkt aus $X$ enspricht genau ein Punkt aus $D^2$.
 Solche Abbildungen kann man sich als Verformung vorstellen: Die $D^2$ sei aus einem Gummiartigen Stoff
 den man nun verdrehen und ziehen kann.
 
 Mit Hilfe von $\Phi$ kann nun jede Selbstabbildung von $X$ zu einer Selbstabbildung von $D^2$ machen.
 
 Folgere, dass  jede stetige Selbstabbildung von $X$ einen Fixpunkt hat.
 Welche geometrischen Figuren $X$ kennt ihr, so dass es eine solche Abbildung gibt. Hier reichen
 Bilder zu den Abbildungen. Ihr m"usst keine expliziten Abbildungen angeben.
 
\end{Aufg}

\begin{proof}
 Falls $f$ eine Selbstabbildung von $X$ ist, so ist $\Phi\inv \circ f \circ \Phi$ eine Selbstabbildung
 von $D^2$, hat also nach dem Browerschen Fixpunktsatz einen Fixpunkt
 $  \Phi\inv \circ f \circ \Phi(x) = x $, was "aquivalent zu $ f( \Phi (x)) = \Phi(x)$. 
 Damit ist $\Phi(x)$ der gesuchte Fixpunkt.
 
 Die geometrischen Figuren sind alle regul"aren Polygone, also Dreick, Viereck etc. Hier reichen 
 Bilder der Abbildungen.
\end{proof}



\begin{Aufg}
Nimm zwei Blatt Papier und lege sie aufeinander. Wir wollen eigentlich eine Selbstabbildung
eines Blatts auf sich selbst betrachten, doch ist es einfacher dafür ein zweites Blatt herzunehmen.
Zerknülle nun das oberere Blatt und lege den Papierball wieder auf das andere Blatt. Nun drücke den 
Ball flach auf das untere Blatt. Jetzt haben wir eine (Selbst)Abbildung eines Blattes auf sich selbst.
Hat diese Abbildung einen Fixpunkt? Ist diese Abbildung eine Kontraktion?
 
\end{Aufg}
\begin{proof}
 Die Abbildung ist stetig, da man beim Zerknüllen das Papier nicht zerreisst.  Damit ist kann man 
 Aufgabe 2 benutzen, und folgern, dass die Abbildung einen Fixpunkt hat. 
 
 Die Abbildung ist keine Kontraktion, da zum Beispiel ein nicht zerknülltes Blatt, also die Identit"at 
 keine Kontraktion ist. Desweiteren ist das zerknüllen lokal die Identit"at, also keine Kontraktion.
 
\end{proof}


\begin{Aufg}
 Zeige das die Gleichungssysteme
 \begin{align*}
  \sin(x+y) - y &=0 \\           
  \cos(x+y) - x &=0
 \end{align*}
 und
  \begin{align*}
  \sin(|x-y|) - y &=0 \\           
  \frac{|x+y|}{4} - x &=0
 \end{align*}
eine L"osung in $\R^2$ besitzen.
\end{Aufg}

\begin{proof}
 
$(x,y)\mapsto (\cos(x+y),\sin(x+y))$ und $(x,y)\mapsto (\frac{|x+y|}{2},\sin(|x-y|))$ k"onnen
auf Selbstabbildungen von $D^2$ auf sich selbst eingeschr"ankt werden. Damit kann man den
Brouwerschen Fixpunktsatz anwenden.
\end{proof}



\begin{Aufg}
 In der Vorlesung wurde definiert was ein Abstandmaß oder Metrik ist:
 \begin{enumerate}[(i)]
  \item $d(x, y) \geq 0 $ und $d(x,y) =0$ genau dann, wenn $x=y$
  \item $d(x,y) = d(y,x)$
  \item $d(x,y) \leq d(x,z) + d(z,y)$ ``Dreiecksungleichung''
 \end{enumerate}
 
 Wir wollen nun nachrechnen ob es überhaupt solche Abbildungen gibt,
 oder ob wir schon solche Abbildungen kennen. 
 \begin{enumerate}
  \item Prüfe ob die Abbildung $(x,y) \mapsto |x-y|$ ein Abstandmaß auf $\R$ ist.
  \item Auf $\R^2$ definiert man $|(x_1, x_2)| \coloneqq \sqrt{x_1 ^2 + x_2 ^2}$.
	Ist dann die Abbildung $(x,y) \mapsto |x-y|$ ein Abstandmaß auf $\R^2$?
  \item Nun wollen wir eine Metrik auf beliebigen Mengen definieren. Sei dazu $X$
	eine Menge. Definiere $\Phi$ durch
  \begin{gather*}
   (a,b)\mapsto \begin{cases}
            1  \text{ wenn } a=b \\
            0  \text{ wenn } a\neq b.
            \end{cases}
  \end{gather*}
 Zeige, dass die Abbildung $(a,b) \mapsto \Phi(a,b)$ ein Abstandmaß auf $X$ definiert. 
 Diese Metrik ist die sogenannte diskrete Metrik.
  
 \end{enumerate}

\end{Aufg}
 
\begin{proof}
 Einfaches nachpr"ufen der Axiome. Dreicksungleichung darf/sollte benutzt werden.
 
\end{proof}

 
\begin{Aufg}
Sei $d(x,y)$ ein Abstandmaß auf einer Menge $X$. Zeige, dass die Abbildung
 $(x,y) \mapsto \frac{d(x,y)}{1 + d(x,y)}$ ein Abstandmaß ist. (Dreicksungleichung ist schwierig)
\end{Aufg}
\begin{proof}
 Nenne die neue Abbildung $g$
 \begin{enumerate}
  \item $g(x,y)= \frac{d(x,y)}{1 + d(x,y)} \geq 0$ da $d(x,y) \geq 0$. Falls $g(x,y) =0$,
  so muss schon $d(x,y)=0$ sein, da $1+d(x,y) >0$.
  \item folgt aus Symmetrie von $d(x,y)$
  \item \begin{align*}
 g(x,z)
 &= \frac{d(x,z)}{d(x,z)+1}
 = 1 - \frac{1}{d(x,z)+1}
 \leq 1 - \frac{1}{d(x,y)+d(y,z)+1} \\
 &= \frac{d(x,y)+d(y,z)}{1+d(x,y)+d(y,z)}
 = \frac{d(x,y)}{1+d(x,y)+d(y,z)} + \frac{d(y,z)}{1+d(x,y)+d(y,z)}\\
 &\leq \frac{d(x,y)}{1+d(x,y)} + \frac{d(y,z)}{1+d(y,z)}
 = g(x,y) + g(y,z)
\end{align*}
 \end{enumerate}

 
\end{proof}



\begin{Aufg}
Im Beweis des Banachschen Fixpunktsatzes wurde eine Formel über die geometrische Reihe
benutzt. Wir wollen diese Aussage beweisen.

Eine geometrische Reihe ist eine Summe der Form $\sum_{k=0}^N p^k$. Sei nun \linebreak $0<p<1$.
Zeige, dass $\sum_{k=0}^N p^k=  \frac{1- p^{N+1}}{1 - p}$.

Betrachte dazu die Ausdrücke 
\begin{gather*}
 \sum_{k=0}^N p^k= s = 1 + p + p^2 + p^3 + \dots + p^N, \text{ und }\\
 ps =p +p^2 +p^3 + p^4 + \dots + p^{N+1}.
\end{gather*}

Falls nun $N$ gegen $\infty$ geht, so konvergiert der Term $\frac{1- p^{N+1}}{1 - p}$
gegen $\frac{1}{1 - p}$, also $\sum_{k=0}^\infty p^k=  \frac{1}{1 - p}$.
Warum ist die Annahme $0<p<1$ wichtig? Würde die Formel in diesen Fällen Sinn ergeben?
\end{Aufg}

\begin{proof}

$s -ps = s(1-p) = 1 + p + p^2 + p^3 + \dots + p^N - (p +p^2 +p^3 + p^4 + \dots + p^{N+1})
= 1 - p^{N+1}$. Daraus folgt $\sum_{k=0}^N p^k = s = \frac{1 - p^{N+1}}{1-p}$. F"ur $p<0$ 
bekommt man eine alternierende Reihe, jedenfalls f"ur $0<|p|<1$. F"ur $p=0$ bekommt man
das Problem, wie $0^0$ definiert ist. F"ur $p=1$ teilt man durch 0. F"ur $1<p$ konvergiert die Reihe 
nicht.
 
\end{proof}


% \begin{Aufg}
%  Sei $g:\R \to \R$ beliebig oft differenzierbar mit $g(\alpha) = \alpha$, und
%  $|g'(\alpha)| <1$. 
%  
%  Zeige, dass es ein $\epsilon > 0$ sodass $g$ das Intervall 
%  $I=\left[ \alpha - \epsilon , \alpha + \epsilon \right]$ auf sich selbst abbildet, d.h. 
%  $g(x)$ ist ein Element von $I$ falls $x$ ein Element von $I$ war.
%  
%  Beweise nun, dass es ein $0<L<1$ gibt, so dass
%  \[
%   |g(x) - g(y)| \leq L|x-y|, \text{ f"ur alle } x,y \text{ in } I.
%  \]
% 
%  Folgere nun, dass die Folge $x_{n+1} = g(x_n)$ f"ur $x_0$ in $I$ gegen $\alpha$
%  konvergiert.
% 
% \end{Aufg}


\begin{Aufg}
 Sei $f: \R \to \R$ eine Abbildung, so dass $g(x) \coloneqq f(f(x))$ eine Kontraktion ist. 
 
 Zeige $f$ hat genau einen Fixpunkt, und der Fixpunkt von $f(x)$ stimmt mit dem von $f(f(x))$
 "uberein.
 
\end{Aufg}

\begin{proof}
 Die Abbildung $g$ ist eine Kontraktion, hat also einen eindeutigen Fixpunkt.
 Ist $\alpha$ ein Fixpunkt von $g$, so ist 
 $ g(f(\alpha)) = f(f(f(\alpha))) = f(g(\alpha))= f(\alpha).$ Da der Fixpunkt von $g$ eindeutig ist,
 ist $\alpha = f(\alpha)$, also auch ein Fixpunkt von $f$. Insbesondere hat $f$ also einen Fixpunkt.
 
 Falls $\alpha$ ein Fixpunkt von $f$ ist, so ist er automatisch ein Fixpunkt von $g$, und damit 
 eindeutig.
\end{proof}



\begin{Aufg}
 Sei $f: \R \to \R$ eine beliebig differenzierbare Funktion, und $\alpha$ eine Nullstelle von $f$,
 mit $f'(\alpha) \neq 0$.
 \begin{enumerate}[(a)]
  \item Zeige: $\alpha$ ist ein Fixpunkt der Funktion $g(x)= x - \frac{f(x)}{f'(x)}$.
  \item Beweise, dass ein $\epsilon > 0$ und $0<L<1$ existiert, so dass $g$ Lipschitz 
  mit Konstante $L$ auf 
  $I=\left[ \alpha - \epsilon , \alpha + \epsilon \right]$ ist, d.h. 
  $|g(x) - g(y)| \leq L|x-y|$ f"ur alle $x,y$ in $I$.
  
  
  \textit{
  Hinweis: Ihr k"onnt ohne Beweis den Mittelwertsatz benutzen: }
  
\textit{  F"ur eine differenzierbare}
  $f:\left[ a, b \right] \to \R$ \textit{existiert ein }$c$ in $\left[ a, b \right]$,
  \textit{so dass}
  $f(b) - f(a) = f'(c)(b-a)$.
  \begin{center}
 \input{Mittelwert.pdf_tex}
 
 \tiny 
 https://commons.wikimedia.org/wiki/File:Mean$\_$value$\_$theorem$\_$(Lagrange's$\_$theorem).svg
\end{center}
 
  
  \item Folgere, dass $g$ eine Selbstabbildung des Intervalls
  $I=\left[ \alpha - \epsilon , \alpha + \epsilon \right]$ ist.
  \item Zeige, dass das Newton-Verfahren konvergiert.
 \end{enumerate}

\end{Aufg}

\begin{proof}\leavevmode
 \begin{enumerate}[(a)]
  \item Ist klar.
  \item Sei $\epsilon>0$ so klein, dass $f'$ auf $\left[ \alpha - \epsilon, \alpha + \epsilon \right]$
   keine Nullstelle hat,  
   \begin{gather*}
   c_\epsilon \coloneqq \max_{x \in \left[\alpha -\epsilon, \alpha +\epsilon \right]} \{ 1, |f(x)| \}, \\
   c' \coloneqq \min_{x \in \left[\alpha - \epsilon, \alpha +\epsilon \right]}\{ 1, |f'(x)| \},\\ 
   c'' \coloneqq \max_{x \in \left[\alpha -\epsilon, \alpha +\epsilon \right]} \{ 1, |f''(x)| \} ,\\
   g'(x)  = 1- \frac{f'(x)f'(x)- f(x)f''(x)}{(f'(x))^2} = \frac{f(x)f''(x)}{(f'(x))^2} 
   \end{gather*}
  Es folgt, dass $g'(\alpha) = 0$, und 
  $|g'(x)| \leq  \frac{c_\epsilon c''}{c'^2}$. Da $f(\alpha)=0$ ist $\lim_{\epsilon
  \to 0} c_\epsilon = 0$. W"ahle nun ein $\delta > 0$, so dass $\frac{c_\delta c''}{c'} \eqqcolon L <1$.
  
  Definiere $a\coloneqq \alpha - \delta, b \coloneqq \alpha + \delta$. Mit dem Mittelwertsatz folgt nun 
  f"ur $x, y$ in $\left[a, b\right]$
  \begin{gather*}
   g(x) - g(y) = g'(c)(x-y), \text{ mit } c \text{ in } \left[a, b\right]. \\
   \Rightarrow|g(x) - g(y)| < L(x-y)
  \end{gather*}

  \item F"ur $x \in \left[a, b\right]$ gilt nun
  \begin{gather*}
   |\alpha- g(x)| = |g(\alpha) - g(x)| < L(\alpha - x) \leq L \cdot \delta < \delta,
  \end{gather*}
  also $g(x) \in \left[a, b\right]$.
  
  \item Das Newton Verfahren erf``ullt die Vorraussetzungen von Aufgabe 9. Der Beweis
  der Konvergenz der 
  Folge $x_{n+1} \coloneqq f(x_n)$, f"ur $x_0 \in \left[a, b\right]$ beliebig, ist der
  gleiche wie im Beweis des Banachschen Fixpunksatzes.

 \end{enumerate}

\end{proof}



% 
% 
% \begin{Aufg}
% 
% Nun wollen wir ein Beispiel für eine Fixpunktiteration betrachten, das sogenannte
% Newton-Verfahren. Sei dazu $f:\R \to \R$
% eine zweimal stetig differenzierbare Funktion mit einer Nullstelle $\alpha$, also $f(\alpha) =0$,
% sodass die Ableitung bei $x_0$ nicht gleich 0 ist. Dann definieren wir die iterierte Folge,
% \[
%  x_{n+1} = x_n - \frac{f(x_n)}{f'(x_n)}, 
% \]
% mit einem Startwert $x_0$. Um die Rechnungen zu vereinfachen, nehmen wir an, dass $x_0$ nahe bei
% der Nullstelle ist, das heißt $|x_0 - \alpha|<1$. 
% Weiter können wir $f$ gut durch seine Ableitung annähern:
% \[
%  f(a) = f(x_n) + f'(x_n)(a-x_n) + C(a-x_n)^2 \text{ für ein } C \in \R.
% \]
% 
% Um zu beweisen, dass die Folge $x_n$ gegen $\alpha$ konvergiert, betrachte die vorherige
% Gleichung mit $a = \alpha$. Dann benutze, dass $f'(x_n) \neq0$, und die Definition der
% Folge $x_n$. Finde damit eine iterative Formel für die Folge des Abstands zu der Nullstelle
% \[
%  \epsilon_n \coloneqq \alpha - x_{n}.
% \]
% Benutze nun, dass $|x_0 - \alpha|<1$ und damit $\epsilon_n$ gegen 0 konvergiert. Damit konvergiert
% $x_n$ gegen $\alpha$.
% 
% \end{Aufg}


\end{document}

