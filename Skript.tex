\documentclass[a4paper,10pt]{article}

\usepackage{mystyle}
\usepackage{amsmath}
\usepackage{amsthm}
\usepackage{mathrsfs} %Für das curly L der linearen Operatoren
\usepackage{empheq}


\begin{document}

\unsure{Hier noch das Tikzbild einfuegen}.

Eine Karte nimmt einen Teil der Welt, zum Beispiel die Bonner Innenstadt,
und bildet diese auf einen anderen Teil der Welt ab. Eine Karte ist also eine
Abbildung der Welt auf sich, welche alles in einen kleineren Teil schickt.

\vspace{.5cm}

{\textbf{Abbildung}}:  $T: X \to X$,  
			          Pkt in der Welt $ \mapsto $Pkt auf Karte
			       $\subset $ Welt.

\vspace{.5cm}		       
Wir wollen einen Fixpunkt dieser Abbildung finden, was bedeutet, dass die Abbildung 
den Punkt nicht ändert.

\vspace{.5cm}		       
 {\textbf{Gesucht}}: Punkt  $x_* \in X$  mit
 $\boxed{T(x_*) = x_*}$ ``Fixpunktgleichung''
\vspace{.5cm}		       
 

 {\textbf{Algorithmus}}: Konstruiere Folge  $(x_n) \in X$  gemäß 
  
    \hspace{2cm} $\boxed{x_{n+1} = T(x_n)}$ ``einfache Fixpunktiteration''
\vspace{.5cm}		       


{\textbf{Beobachtung}}: $T$ ist stetig, d.h. $x_n \to x$ impliziert $T(x_n) \to T(x)$.

\vspace{.5cm}		       

{\textbf{Folgerung}}: Wenn wir zeigen können, dass die Folge $(x_n)$ konvergiert,
so ist der Grenzwert ein geuschter Fixpunkt, denn
\[
 \lim_{n \to \infty} x_n = x \Rightarrow T(x_*) = T(\lim_{n \to \infty} x_n) =
 \lim_{n \to \infty} T(x_n) = \lim_{n \to \infty} x_{n+1} = x_*.
\]


\textbf{Wichtige Beobachtung}: $T$ ist eine \textit{Kontraktion}, d.h. es gilt
\[
 \text{Abstand}(T(x), T(y)) \leq L \cdot \text{Abstand}(x, y)
\]

mit $L < 1$ (Kartenmaßstab).


\begin{thrm*}[Banachscher Fixpunktsatz (1922)]

Ist $X$ ein vollständiger, metrischer Raum, und $T: X\to X$ eine Kontraktion, so besitzt $T$
\underline{genau einen} Fixpunkt $x_* \in X$ und die durch $x_{n+1} = T(x_n)$ konstruierte Folge
konvergiert gegen diesen Fixpunkt für jeden Startwert $x_0 \in X$.
\end{thrm*}
Dies wollen wir beweisen. Dazu einige strenge Definitionen.

\begin{defi}
 Eine Menge $X$, auf welche ein \underline{Abstandsmaß} $d: X \times X \to \R$ defniert ist,
 heißt metrischer Raum falls
 \begin{enumerate}
  \item $d(x, y) \geq 0 $ und $d(x,y) =0$ genau dann, wenn $x=y$
  \item $d(x,y) = d(y,x)$
  \item $d(x,y) \leq d(x,z) + d(z,y)$ ``Dreiecksungleichung''
 \end{enumerate}

\end{defi}

\begin{defi}
 Ein metrischer Raum heißt \underline{vollständig}, wenn jede Cauchyfolge konvergiert.
 Eine Folge $(x_n)$ heißt \underline{Cauchyfolge} fall es zu jedem $\epsilon > 0$ ein 
 $N_\epsilon \in \N$ gibt, sodass
 \[
  d(x_n, x_m) < \epsilon \text{ für ale } m,n \geq N_\epsilon.
 \]

\end{defi}

\begin{expl*}
 Der Zahlenstrahl $\R$ ist vollständig bzgl. Abstandsmaß $d(x,y)= |x - y|$.
 
 \noindent
 Die Ebene $\R^2$ ist vollständig bzgl. Abstandsmaß $d(x,y) = \sqrt{(x_1 - y_1)^2 + (x_2 - y_2)^2}$.
\end{expl*}

\underline{Frage} :Ist die Welt vollständig? Dies ist eine Frage der Philosophie der Mathematik.

Nun aber zurück zum Beweis des Fixpunktsatzes.

\vspace{0.5cm}
\noindent
\textbf{Beweis (Banachscher Fixpunktsatz):}

 \vspace{.2cm}
 \noindent
 \textbf{1) Eindeutigkeit}:
 Gäbe es zwei Fixpunkte $x_* \neq y_*$, so w"are \newline
 \[d(x_*, y_*) = d(T(x_*), T(y_*)) \leq L \cdot d(x_*, y_*)\]
 was wegen $L< 1$ $d(x_*,y_*) =0$ nach sich zieht, also $x_*= y_*$.

 \vspace{.2cm}
 \noindent
 \textbf{2) Existenz}: 
 Wir zeigen, dass $(x_n)$ eine Cauchyfolge ist. Sei dazu $\epsilon > 0$ beliebig aber fest.
 Zunächst ist 
 \[
  d(x_{n+1}, x_n) = d(T(x_n), T(x_{n-1})) \leq L\cdot d(x_n, x_{n-1})\leq \dots \leq L^n d(x_1, x_0).
 \]
F"ur beliebige $m > n$ folgt dann mit der Dreiecksungleichung
\begin{align*}
 d(x_m, x_n) & \leq d(x_m, x_{m-1}) + d(x_{m-1}, x_n) \\
	     & \leq d(x_m, x_{m-1}) + d(x_{m-1}, x_{m-2}) +d(x_{m-2}, x_n) \\
	     & \leq \dots \\
	     & \leq d(x_m, x_{m-1}) + d(x_{m-1}, x_{m-2})+ \dots +d(x_{n+1}, x_n) \\
	     & \leq L^{m-1} d(x_1, x_0) + L^{m-2}d(x_1, x_0) + \dots L^nd(x_1, x_0) \\
	     & = (L^{m-1} + L^{m-2}+ \dots L^{n})d(x_1, x_0) \\
	     & = L^n(1 + L + L^2 + \dots + L^{m-1-n})d(x_1, x_0) \\
	     & \leq L^n \left(\frac{1}{1-L}\right)d(x_1, x_0).
\end{align*}
Die letzte Ungleichung kommt durch die geometrische Reihe zustande. Wir wollen beweisen, dass
$(x_n)$ eine Cauchyfolge ist. Per Definition m"ussen wir beweisen, dass es f"ur jedes
$\epsilon > 0$ ein 
 $N_\epsilon \in \N$ gibt, sodass
 \[
  d(x_n, x_m) < \epsilon \text{ für ale } m,n \geq N_\epsilon.
 \]
Dazu bemerken wir, dass der rechte Faktor der letzten Zeile $\left(\frac{1}{1-L}\right)d(x_1, x_0)$ 
eine feste positive Zahl ist. Obwohl $ d(x_m, x_n)$ sowohl von $m$ und $n$ abh"angig ist, konnten
wir den Term gegen einen Term absch"atzen der nur noch von $n$ abh"angt. Man bemerke, dass die 
Vorraussetzung $m>n$ hier entscheidend war. Der Term $L^n$ wird f"ur große $n$ immer kleiner, da 
$L<1$. Insbesondere wird $L^n < \frac{\epsilon}{\left(\frac{1}{1-L}\right)d(x_1, x_0)}$, was beweist,
dass $(x_n)$ eine Cauchyfolge ist.

Da $X$ vollständig ist, existiert ein Grenzwert $\lim_{n \to \infty} x_n = x_*$. Wir haben schon in 1)
gezeigt, dass $x_*$ ein Fixpunkt ist. Damit ist der Beweis beendet. $\qed$

\begin{rem*}
 \underline{Konvergenzgeschwindigkeit} $d(x_n, x_*) \leq L^n \cdot$ const.
\end{rem*}

\noindent
\textbf{Anwendung von Fixpunktiterationen}

L"osung nichtlinearer Gleichungen
\[
 f(x)=0 \Leftrightarrow x = x + f(x) \Leftrightarrow 
 x = x + \frac{f(x)}{f'(x)} \text{ falls } f'(x)\neq0.
\]
Fixpunktiteration: $x_{n+1} = x_n - \frac{f(x_n)}{f'(x_n)}$

Sind Kontraktionen wirklich notwendig?

\begin{thrm*}
 Sei $f:\left[ 0, 1\right] \to \left[ 0, 1\right]$ stetig. Dann besitzt $f$ mindestens 
 einen Fixpunkt.
\end{thrm*}
\unsure{Beweisskizze?}

\noindent
Unterschiede zum Banachschen Fixpunktsatz:
\begin{itemize}
 \item[$+$] keine Kontraktionsannahme
 \item[$-$] keine Eindeutigkeit von Fixpunkten
 \item[$-$] kein Verfahren zur Bestimmung des Fixpunktes
\end{itemize}
\unsure{Animationen}

\noindent
\textbf{Fazit}: schw"achere Annahmen $\Rightarrow$ schw"achere Resultate

\begin{thrm*}[Browerscher Fixpunktsatz (1910)]
 Sei $K$ die abgeschlossene Kugel vom Radius 1 in $\R^2$ oder $\R^3$. Jede stetige
 Funktion $f:K \to K$ besitzt mindestens einen Fixpunkt
\end{thrm*}

Algorithmische Bestimmung eines solchen Fixpunktes ist ungleich schwieriger.









\end{document}
